\section{Zusammenfassung}


% Für jede Folie eine Frame-Umgebung erstellen
% Innerhalb der Frame-Umgebung werden dann die Inhalte geschrieben


\begin{frame}
	\frametitle{Platzhalter}

	% Define block styles
	\tikzstyle{decision} = [diamond, draw, fill=RSTgreen!50, 
	    text width=4.0em, text badly centered, node distance=2cm, inner sep=0pt]
	\tikzstyle{block} = [rectangle, draw, fill=RSTgreen!50, 
	    text width=5em, text centered, rounded corners, minimum height=2em]
	\tikzstyle{line} = [draw, -latex']
	\tikzstyle{cloud} = [draw, ellipse,fill=RSTorange!60, node distance=3cm, minimum height=2em]
	    
	\begin{center}
	\begin{tikzpicture}[node distance = 1.4cm, auto, every node/.style={font=\sffamily\scriptsize}]
	    % Place nodes
	    \node [block] (init) {initialize model};
	    \node [cloud, left of=init] (expert) {expert};
	    \node [cloud, right of=init] (system) {system};
	    \node [block, below of=init] (identify) {identify candidate models};
	    \node [block, below of=identify] (evaluate) {evaluate candidate models};
	    \node [block, left of=evaluate, node distance=3cm] (update) {update model};
	    \node [decision, below of=evaluate] (decide) {is best candidate better?};
	    \node [block, below of=decide, node distance=1.9cm] (stop) {stop};
	    % Draw edges
	    \path [line] (init) -- (identify);
	    \path [line] (identify) -- (evaluate);
	    \path [line] (evaluate) -- (decide);
	    \path [line] (decide) -| node [near start] {yes} (update);
	    \path [line] (update) |- (identify);
	    \path [line] (decide) -- node {no}(stop);
	    \path [line,dashed] (expert) -- (init);
	    \path [line,dashed] (system) -- (init);
	    \path [line,dashed] (system) |- (evaluate);
	\end{tikzpicture}
	\end{center}
 		 		
\end{frame}

\begin{frame}{Weitere Beispiele}
	\setbeamercovered{transparent} % invisible / transparent
	\begin{bitemize}
		\item Dies ist das erste Item.
		\item Dies ist das zweite Item.
		\item Dies ist das dritte Item.
	\end{bitemize}
	
	\setbeamercovered{invisible}
	\pause
	Es gibt auch einen Pausebefehl!

\end{frame}

%\begin{frame}{Tikz animation example}
%	\begin{center}
%	\begin{animateinline}[controls=false,autoplay,palindrome,loop]{10}
%		\multiframe{50}{nYinc=-0.5+0.02, nAngleA=0+0.5, nAngleB=-180+0.5, nColor=0+2}{
%		\makebox[0.8\textwidth]{
%			\begin{tikzpicture}
%				\tikzstyle{circlestyle}=[gray];
%				\tikzstyle{trajectory}=[shorten <= 0.1cm, shorten >= 0.1cm];
%				\node [rectangle, minimum width= 5cm, minimum height = 3cm, anchor = center] at (0,0) {};
%				\filldraw[circlestyle] (-3,0) circle (0.1cm) coordinate (start) node[above] {\small Start};
%				\filldraw[circlestyle] (3,0) circle (0.1cm) coordinate (goal) node[above] {\small Goal};
%				\node[anchor = south] (top) at (0, 0.5) {\small $\tau_1$};
%				\coordinate (middle) at (0,\nYinc);
%				\node[anchor = south] (bottom) at (0,-0.5) {\small $\tau_2$};
%				\draw[trajectory, line width=0.5mm, color=RSTgreen] (start) to[out=0, in=-155]  (top.south) to[out=25, in=180] (goal);
%				\draw[trajectory, line width=0.4mm, color=RSTgreen!\nColor!RSTorange] (start) to[out=0, in=\nAngleB] (middle) to[out=\nAngleA, in=180] (goal);
%				\draw[trajectory, line width=0.5mm, color=RSTorange] (start) to[out=0, in=180] (bottom.south) to[out=0, in=180] (goal);]
%			\end{tikzpicture}
%			}
%		}
%	\end{animateinline}
%	\end{center}
%\end{frame}

%\begin{frame}[fragile]{Video}
%
%	\begin{center}
%		\includemedia[
%			width=0.6\linewidth,keepaspectratio,
%			addresource=Videos/FILENAME.mp4,
%			transparent, %transparent player background
%			activate=onclick, % pageopen or onclick
%			passcontext, %show VPlayer’s right-click menu
%			flashvars={
%				source=VIDEO_PATH
%				&loop=true % loop video
%				&autoPlay=true
%			}
%		]{\includegraphics{PREVIEW_IMAGE_PATH}}{VPlayer.swf}
%	\end{center}
%
%   BUT ITS'S SMARTER TO USE OUR FANCY MACRO:
%%	\begin{center}
%%		\placeVideo[onclick]{VIDEO_PATH}{PREVIEW_IMAGE_PATH}{0.6\linewidth} % onclick or pageopen
%%	\end{center}
%%  This macro is able to detect, if the package media9 was loaded properly, otherwise only the preview image will be displayed.
%
%
%\end{frame}
